
%%% Local Variables:
%%% mode: latex
%%% TeX-master: t
%%% End:

% !Mode:: "TeX:UTF-8"

\chapter{引言}
\label{cha:intro}


\section{研究背景与相关科学问题}
当前,面向复杂应用的高性能片上系统为了规避和减轻功耗墙(Power Wall)问题\onlinecite{ThouCrCpTechPer},延续摩尔定律\onlinecite{MooreRule},采用了两种主要的技术手段。
首先,必须在芯片运行中,通过合理任务调度来降低芯片的运行能耗和峰值工作温度。 因此,对芯片进行实时功耗温度管理(DPTM)的算法研究就具有重要的理论意义与广阔的应用前景, 是目前电子设计自动化(EDA)研究的一个热点问题。
最初为了降低芯片运行功耗、延长设备电池的使用寿命, 研究人员提出了运用动态电压调节技术(DVS)对系统动态功耗进行实时功耗管理(DPM)\onlinecite{TheoPracLimofDVS, LkDVSRTEmbSys, TemSchAsgnHdRTApp, TheIntMtLPICDigtDsg}。 然而,随着IC进入纳米工艺,漏电流静态功耗已经超过动态功耗,成为芯片功耗的主要来源, 而且漏电流和工作温度之间存在指数关系\onlinecite{TkSchAlgRTLkPowTemOpt, BSIM, TemMngSysHiPerfPPC},如对于65nm工艺,当温度从60摄氏度增加到80摄氏度,芯片漏电流会增加21\%。
其次,目前IC业界已经普遍采用多核并行计算结构来提升芯片性能(通量)、降低设计复杂度。 采用多核并行计算架构的多核片上系统(MPSoC)带来了热点分散的问题,即每个核都会产生一个局部热点\onlinecite{MnyDsgThemPer}。 为了将MPSoC多个热点的温度控制在一个安全阈值内,必须在设计与运行阶段,以功能模块与处理器核为单位, 对芯片的功耗分布\onlinecite{ThemOptMulMulFlp, FreqVolPlMulProThemConst, ThmBefMulCrFlrPlLimSty}与任务调度\onlinecite{ThrOptTskAllocThemConstMulPro, TskAllocMinSysPowHomoMulPro, ThemOlTskAllocMulProThrOpt}进行优化, 为此需要在结构级对芯片进行快速准确的热分析\onlinecite{ThmBefMulCrFlrPlLimSty,ThrOptTskAllocThemConstMulPro, LrnAutoRegMdlFstTransThemAlysCp}。 鉴于纳米工艺CMOS器件的漏电流随着工作温度的升高而指数增加,漏电流功耗与温度之间存在直接的依赖关系, 即电热耦合效应\onlinecite{ThrOptTskAllocThemConstMulPro}。 为了提高分析的精度,必须在结构级热分析方法研究中考虑电热耦合效应\onlinecite{ThrOptTskAllocThemConstMulPro, TskAllocMinSysPowHomoMulPro, ThemOlTskAllocMulProThrOpt}。

\section{已有研究成果及其缺陷}
针对任务调度领域,研究人员开始针对微处理器和大型服务器系统进行实时温度管理(DTM)\onlinecite{CtrlTechThemRCMdlAccDTM, TemMicArcMdlImpl, PreDTMMulApp}。 为了对片上系统进行功耗、温度的统一调度与管理,最近开始出现了实时功耗温度管理(DPTM)的研究报道\onlinecite{TemLkMinTechRTSys, EngRTTskSchTemDepLk, TemIdDistEngOptDVS, LkEngMinRTSysMaxTemConst}, 在考虑漏电流、温度相互作用关系和实时任务的时间限制这两个前提下,采用不同的DPTM策略来达到最小化运行能耗的目的。
在DPTM研究中,为了提高DPTM系统的降温降耗效果,必须对系统的任务负载进行精确的预测, 事实上,任务负载的轻重决定了不同方法的DPTM效果。
对于多核芯片的热分析,受惠于电热分析的相似性,可以采用有限差分方法(PDF)可以进行全芯片三维热分析, 获得温度分布的精确解\onlinecite{ElecAlysOptTechNanoIC}; 为了考虑温度对功耗的影响,可以采用迭代方法来逼近最后的精确解\onlinecite{HotSpotCmptThemMdlMethErVLSIDsn}。 基于PDF求解的HotSpot是目前广泛采用的热分析工具软件,能够用于MPSoC的结构级热分析, 也能够对电热耦合效应进行求解\onlinecite{HotSpotCmptThemMdlMethErVLSIDsn}。 尽管PDF方法可以获得高精度的求解方案,但这类方法的算法复杂度非常高, 不满足MPSoC布图规划和实时功耗温度管理对结构级快速求解的需求\onlinecite{ThmBefMulCrFlrPlLimSty,LrnAutoRegMdlFstTransThemAlysCp}。
为了对结构级设计的温度分布进行快速求解, 出现过多种加速算法\onlinecite{MnyDsgThemPer,MySSTAPaper,ThemOptMulMulFlp,ThrOptTskAllocThemConstMulPro, TskAllocMinSysPowHomoMulPro,LrnAutoRegMdlFstTransThemAlysCp}。 文献\onlinecite{ThemOptMulMulFlp}采用最简单的物理距离模型,速度速度最快、精度最差,无法进行精确的MPSoC温度求解。 文献\onlinecite{MnyDsgThemPer,ThrOptTskAllocThemConstMulPro,TskAllocMinSysPowHomoMulPro}省略了核间的侧向热阻、来简化温度求解, 其优点是速度快,缺点是降低了求解的精度。 文献\onlinecite{HotCrThemCoupFtrMulArch}采用基于学习的自回归算法进行在线温度分析,提高热分析速度的同时、也降低了求解的精度。 总之,求解加速的代价是降低了求解的精度。
为了考虑温度对功耗的影响(LDT),精确的求解算法必须采用迭代的方法进行逼近求解\onlinecite{ThmBefMulCrFlrPlLimSty}。 在现有结构级热分析算法中,为了提高求解速度,文献\onlinecite{MnyDsgThemPer}没有考虑LDT, 文献\onlinecite{TskAllocMinSysPowHomoMulPro}采用线性模型来拟合LDT, 文献\onlinecite{ThrOptTskAllocThemConstMulPro,ThemOlTskAllocMulProThrOpt}采用分段拟合系数矩阵来求解LDP效应, 其结果会带来求解精度不同程度的降低。

\section{本文工作及其贡献}
为了弥补上文指出的已有研究的不足之处,本文对温度敏感的实时功耗调度和多核芯片的热分析方法这两个不同领域,分别做了较为深入的研究, 并取得了如下成果。
首先,为了构建一个高效的DPTM系统,本文不仅提出了一种具有高精度的组合式任务预测方法, 而且还提出一种新的DPTM任务调度算法VP-TALK,并进一步集成了一个基于负载预测的DPTM原型系统,该系统主要包括工作负载预测、 任务实时调度两大模块。
(1)基于组合任务预测方法的负载预测模块:根据频率范围,先将对应于复杂应用的任务分为随机/周期/趋势三种组分, 然后采用灰色模型/傅里叶模型/RBF神经网络模型分别对这三种组分进行精确分析,最后将三部分预测结果合成为复杂任务的预测值。
(2)基于多种调度算法的实时调度模块:先根据对工作负载率的精确预测值、计算出最优工作状态的电压/频率理想值, 再从系统的电压/频率对的实际设定值中选取相邻的两个工作状态,最后考虑系统实时性、温度上限限制、 静态功耗与温度的敏感关系以及芯片模式切换代价等多种因素, 利用机器学习的方法,选择一种最优的调度策略。
大量的模拟实验表明,(1)在负载预测方面,本文DPTM系统所采用的组合任务预测方法胜过众多的相关模型及算法,平均误差仅为2.89\%; (2)在节能效果方面,当负载率高于55\%时,基于相同的峰值温度约束, 本文所提出的VP-TALK算法分别比Pattern-based、M-oscillating和TALK对比算法节能约20.5\%、11.0\%、11.5\%; (3)本文DPTM原型系统的调度效果接近于理想调度效果。
其次,本文采用自下而上的策略,使用HotSpot提取MPSoC功能模块之间的热相关系数,建立了模块级热分析方法BloTAM; 如图2所示,每个核内只产生一个热点,我们可以仅依靠热点之间的热相关系数、建立一个算法复杂度非常低的核级热分析方法CorTAM; 为了提高CorTAM的精度,我们进一步提出了考虑本核内模块相互影响的改良核级方法BiCorTAM。与现有的结构级热分析算法相比, 本文所提出的三种方法均具有简单、高效、与现有简化模型兼容、易于扩展、考虑LDT影响等优点,可以满足温敏MPSoC设计对高效、 精确的结构级热分析方法的需求。
与HotSpot软件的实验结果相比,本文方法的实验数据表明:(1)对核数较多MPSoC进行局部热点温度分析的时候, BloTAM和BiCorTAM只产生2\%、3\%以下的温度增量平均误差,是高精度的结构级热分析方法。 (2)在采用电压频率调节的温敏16核CPU布图规划研究中,在包含参数提取时间的情况下,BloTAM和BiCorTAM可以提供50倍左右的计算加速。 (3)从总体效果来看,在本文所提出三种建模分析方法中,BloTAM和BiCorTAM方法可以提供满意的分析精度与计算加速, 是较为理想的MPSoC结构级热分析方法。

\section{文章结构安排}
文章结构安排如下:第二章介绍基于高精度组合式任务预测方法的DPTM原型系统,研究对象仅限于单一处理器。 第四章给出大量模拟实验数据,以证实该调度系统在降低功耗和温敏控制上的优越性。第四章将研究对象扩展为多核处理器, 提出三种结构级热分析方法。 为衡量这三种热分析计算模型的精确度与加速效果,第五章中设计了若干实验测例,并给出了模拟热分析结果。最后一章对全文作出总结。 